%-----------------------------------------------------
% index key words
%-----------------------------------------------------
\index{solution set}

%-----------------------------------------------------
% name, leave blank
% title, if the exercise has a name i.e. Hilbert's matrix
% difficulty = n, where n is the number of stars
% origin = "by name \cite{ref}"
%-----------------------------------------------------
\begin{Exercise}[
name={},
title={}, 
difficulty=0,
origin={\cite{KK}}]
If a system of linear equations has fewer equations than variables and
there exist a solution to this system. Is it possible that
your solution is the only one?\ Explain.
\end{Exercise}
\begin{Answer}
No.  There must be a free variable and since the system is consistent there are infinitely many solutions 
\end{Answer}
