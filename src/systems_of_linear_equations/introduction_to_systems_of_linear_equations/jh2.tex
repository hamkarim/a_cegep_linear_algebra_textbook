%-----------------------------------------------------
% index key words
%-----------------------------------------------------
\index{geometric interpretation}


%-----------------------------------------------------
% name, leave blank
% title, if the exercise has a name i.e. Hilbert's matrix
% difficulty = n, where n is the number of stars
% origin = "by name \cite{ref}"
%-----------------------------------------------------
\begin{Exercise}[
name={},
title={}, 
difficulty=0,
origin={\cite{JH}}]
In the system
\begin{equation*}
\begin{linsys}{2}
ax  &+  &by  &=  &c  \\
dx  &+  &ey  &=  &f  
\end{linsys}
\end{equation*}
each of the equations describes a line in the \( xy \)-plane.
By geometrical reasoning, show that there are three possibilities:
there is a unique solution, there is no solution, 
and there are infinitely many solutions.
\end{Exercise}

\begin{Answer}
     Recall that if a pair of lines share two distinct points then
      they are the same line. 
      That's because two points determine a line, so these
      two points determine each of the two lines, 
      and so they are the same line.

      Thus the lines can share one point (giving a unique solution), 
      share no points (giving no solutions), or
      share at least two points (which makes them the same line). 
\end{Answer}
