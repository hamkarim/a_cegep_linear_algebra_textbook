%-----------------------------------------------------
% index key words
%-----------------------------------------------------
\index{subspace}
\index{spaning set}
\index{matrix space, polynomial space}

%-----------------------------------------------------
% name, leave blank
% title, if the exercise has a name i.e. Hilbert's matrix
% difficulty = n, where n is the number of stars
% origin = "\cite{ref}"
%-----------------------------------------------------
\begin{Exercise}[
name={},
title={}, 
difficulty=0,
origin={\cite{JH}}]
Express each subspace as a span of a set of vectors.
      
\Question  \( \set{\rowvec{a &b &c}\suchthat a-c=0}   \)
      
\Question \(
          \set{\begin{mat}
                 a  &b  \\
                 c  &d
               \end{mat}  \suchthat a+d=0}
     \) 
\Question \(
          \set{\begin{mat}
                 a  &b  \\
                 c  &d
               \end{mat}  \suchthat \text{\( 2a-c-d=0 \) 
                                                 and \( a+3b=0 \)} }
        \)
      
\Question \( \set{a+bx+cx^3\suchthat a-2b+c=0} \)
      
\Question The subset of \( \polyspace_2 \) of quadratic polynomials 
        \( p \) such that \( p(7)=0 \)
\end{Exercise}

\begin{Answer}
\Question \( \set{\rowvec{c &b &c}\suchthat b,c\in\Re}
                      =\set{b\rowvec{0 &1 &0}+c\rowvec{1 &0 &1}
                        \suchthat b,c\in\Re} \)
           The obvious choice for the set that spans is 
           $\set{\rowvec{0 &1 &0},\rowvec{1 &0 &1}}$.
\Question \( \set{\begin{mat}
                       -d &b  \\
                       c  &d
                      \end{mat} \suchthat b,c,d\in\Re}
                     =\set{b\begin{mat}[r]
                       0  &1  \\
                       0  &0
                      \end{mat}
                     +c\begin{mat}[r]
                       0  &0  \\
                       1  &0
                      \end{mat}
                     +d\begin{mat}[r]
                       -1  &0  \\
                       0  &1
                      \end{mat}  \suchthat b,c,d\in\Re} \)
            One set that spans this space consists of those three matrices. 
\Question The system
          \begin{equation*}
            \begin{linsys}{4}
              a  &+  &3b  &   &   &  &  &=  &0  \\
             2a  &   &    &   &-c &- &d &=  &0  
            \end{linsys}
          \end{equation*}
          gives \( b=-(c+d)/6 \) and \( a=(c+d)/2 \).
          So one description is this.
          \begin{equation*}
            \set{c\begin{mat}[r]
                       1/2  &-1/6  \\
                       1    &0
                      \end{mat}
                     +d\begin{mat}[r]
                       1/2  &-1/6  \\
                       0    &1
                      \end{mat}  \suchthat c,d\in\Re}
           \end{equation*}
          That shows that a set spanning this subspace consists of those
          two matrices.
\Question The $a=2b-c$ gives that the set
           \( \set{(2b-c)+bx+cx^3 \suchthat b,c\in\Re} \)
           equals the set
           \( \set{b(2+x)+c(-1+x^3) \suchthat b,c\in\Re}  \).
           So the subspace is the span of the set $\set{2+x,\; -1+x^3}$.
\Question The set
          \( \set{a+bx+cx^2\suchthat a+7b+49c=0} \)
          can be parametrized as
          \begin{equation*}
            \set{b(-7+x)+c(-49+x^2)\suchthat b,c\in\Re} 
          \end{equation*}
          and so 
          has the spanning set $\set{-7+x,\;-49+x^2}$.

\end{Answer}
