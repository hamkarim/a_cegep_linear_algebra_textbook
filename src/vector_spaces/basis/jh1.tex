%-----------------------------------------------------
% index key words
%-----------------------------------------------------
\index{polynomial space}
\index{basis}

%-----------------------------------------------------
% name, leave blank
% title, if the exercise has a name i.e. Hilbert's matrix
% difficulty = n, where n is the number of stars
% origin = "\cite{ref}"
%-----------------------------------------------------
\begin{Exercise}[
name={},
title={}, 
difficulty=0,
origin={\cite{JH}}]
Determine if each is a basis for $\polyspace_2$.
\Question $\sequence{x^2-x+1,\; 2x+1,\; 2x-1}$
\Question $\sequence{x+x^2,\; x-x^2}$
\end{Exercise}

\begin{Answer}
\Question This is a basis for $\polyspace_2$.
           To show that it spans the space we consider a generic
           $a_2x^2+a_1x+a_0\in\polyspace_2$ and look for 
           scalars $c_1, c_2, c_3\in\Re$ such that
           $a_2x^2+a_1x+a_0=c_1\cdot(x^2-x+1) +c_2\cdot(2x+1) +c_3(2x-1)$.
           Gauss's Method on the linear system
           \begin{equation*}
             \begin{linsys}{3}
               c_1  &\spaceforemptycolumn  &     &  &     &=  &a_2 \\
                    &  &2c_2 &+ &2c_3 &=  &a_1 \\
                    &  &c_2  &- &c_3   &=  &a_0
             \end{linsys}
           \end{equation*}
           shows that given the $a_i$'s we can compute the $c_j$'s as
           $c_1=a_2$, $c_2=(1/4)a_1+(1/2)a_0$, and $c_3=(1/4)a_1-(1/2)a_0$.
           Thus each element of $\polyspace_2$ is a combination of the 
           given three.

           To prove that the set of the given three is linearly independent
           we can set up the equation 
           $0x^2+0x+0=c_1\cdot(x^2-x+1) +c_2\cdot(2x+1) +c_3(2x-1)$
           and solve, and it will give that $c_1=0$, $c_2=0$, and $c_3=0$.
\Question This is not a basis.
           It does not span the space since no combination of the two
           $c_1\cdot (x+x^2) +c_2\cdot (x-x^2)$ will sum to the polynomial 
           $3\in\polyspace_2$.  
\end{Answer}
