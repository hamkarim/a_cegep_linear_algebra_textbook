%-----------------------------------------------------
% index key words
%-----------------------------------------------------
\index{basis}
\index{polynomial space}

%-----------------------------------------------------
% name, leave blank
% title, if the exercise has a name i.e. Hilbert's matrix
% difficulty = n, where n is the number of stars
% origin = "\cite{ref}"
%-----------------------------------------------------
\begin{Exercise}[
name={},
title={}, 
difficulty=0,
origin={\cite{JH}}]
Find one vector $\vec{v}$ that will make each into a basis
for the space.
\Question $\sequence{\colvec[r]{1 \\ 1},\vec{v}}$ in $\Re^2$
\Question $\sequence{\colvec[r]{1 \\ 1 \\ 0},
                            \colvec[r]{0 \\ 1 \\ 0},\vec{v}}$ in $\Re^3$
\Question $\sequence{x,\;1+x^2,\;\vec{v}}$ in $\polyspace_2$
\end{Exercise}

\begin{Answer}
      Each forms a linearly independent set if we omit $\vec{v}$.
      To preserve linear independence, we must expand the span of each.
      That is, we must determine the span of each (leaving $\vec{v}$ out),
      and then pick a $\vec{v}$ lying outside of that span.
      Then to finish, we must check that the result spans the entire given
      space.
      Those checks are routine.
      \Question Any vector that is not a multiple of the given one, 
          that is, any vector that is not on the line $y=x$ will do here.
          One is $\vec{v}=\vec{e}_1$.
      \Question By inspection, we notice that the vector $\vec{e}_3$ is
          not in the span of the set of the two given vectors.
          The check that the resulting set is a basis for $\Re^3$ is 
          routine.
      \Question For any member of the span 
          $\set{c_1\cdot(x)+c_2\cdot(1+x^2)\suchthat c_1,c_2\in\Re}$,
          the coefficient of $x^2$ equals the constant term.
          So we expand the span if we add a quadratic without this property,
          say, $\vec{v}=1-x^2$.
          The check that the result is a basis for $\polyspace_2$ is easy.  
\end{Answer}
