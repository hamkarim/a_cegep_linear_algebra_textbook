%-----------------------------------------------------
% index key words
%-----------------------------------------------------
\index{basis}

%-----------------------------------------------------
% name, leave blank
% title, if the exercise has a name i.e. Hilbert's matrix
% difficulty = n, where n is the number of stars
% origin = "\cite{ref}"
%-----------------------------------------------------
\begin{Exercise}[
name={},
title={}, 
difficulty=0,
origin={\cite{JH}}]
    Where
    \( \sequence{\vec{\beta}_1,\dots,\vec{\beta}_n } \)
    is a basis, show that in this equation
    \begin{equation*}
       c_1\vec{\beta}_1+\dots+c_k\vec{\beta}_k
       =
       c_{k+1}\vec{\beta}_{k+1}+\dots+c_n\vec{\beta}_n
    \end{equation*}
    each of the \( c_i \)'s is zero.
    Generalize.
\end{Exercise}

\begin{Answer}
      To show that each scalar is zero, simply subtract
      \( c_1\vec{\beta}_1+\dots+c_k\vec{\beta}_k
          -c_{k+1}\vec{\beta}_{k+1}-\dots-c_n\vec{\beta}_n=\zero \).
      The obvious generalization is that in any equation involving only the
      \( \vec{\beta} \)'s, and in which each \( \vec{\beta} \) appears only
      once, each scalar is zero.
      For instance, an equation with a combination of 
      the even-indexed basis vectors
      (i.e., $\vec{\beta}_2$, $\vec{\beta}_4$, etc.) on the right and the
      odd-indexed basis vectors on the left also gives the conclusion that
      all of the coefficients are zero. 
\end{Answer}
