%-----------------------------------------------------
% index key words
%-----------------------------------------------------
\index{vector space}
\index{additive inverse}

%-----------------------------------------------------
% name, leave blank
% title, if the exercise has a name i.e. Hilbert's matrix
% difficulty = n, where n is the number of stars
% origin = "\cite{ref}"
%-----------------------------------------------------
\begin{Exercise}[
name={},
title={}, 
difficulty=0,
origin={\cite{JH}}]
In a vector space every element has an additive inverse.
Is the additive inverse unique \textit{(Can some elements have two or more)}?
\end{Exercise}

\begin{Answer}



Each element of a vector space has one and only one additive
inverse.

For, let \( V \) be a vector space and suppose that \( \vec{v}\in V \).
If \( \vec{w}_1,\vec{w}_2\in V \) are both additive inverses of
\( \vec{v} \) then consider \( \vec{w}_1+\vec{v}+\vec{w}_2 \).
On the one hand, we have that it equals $\vec{w}_1+(\vec{v}+\vec{w}_2)=
\vec{w}_1+\zero=\vec{w}_1$.
On the other hand we have that it equals $(\vec{w}_1+\vec{v})+\vec{w}_2=
\zero+\vec{w}_2=\vec{w}_2$.
Therefore, $\vec{w}_1=\vec{w}_2$.

\end{Answer}
