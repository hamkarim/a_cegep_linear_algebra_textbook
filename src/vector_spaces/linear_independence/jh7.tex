%-----------------------------------------------------
% index key words
%-----------------------------------------------------
\index{linear independence}
\index{linear dependence}

%-----------------------------------------------------
% name, leave blank
% title, if the exercise has a name i.e. Hilbert's matrix
% difficulty = n, where n is the number of stars
% origin = "\cite{ref}"
%-----------------------------------------------------
\begin{Exercise}[
name={},
title={}, 
difficulty=0,
origin={\cite{JH}}]
\Question When is a one-element set linearly independent?
\Question When is a two-element set linearly independent?
\end{Exercise}

\begin{Answer}
\Question A singleton set $\set{\vec{v}}$ is linearly independent 
          if and only if $\vec{v}\neq\zero$.
          For the `if' direction, with $\vec{v}\neq\zero$, 
          we consider the relationship
          \( c\cdot\vec{v}=\zero \) and noting that the only solution
          is the trivial one:~$c=0$.
          For the `only~if' direction, it is evident from the definition. 
\Question A set with two elements is linearly independent 
          if and only if neither member is a  multiple of the other 
          (note that if one is the zero vector then it is a multiple of the
          other).
          This is an equivalent statement:~a set is linearly dependent if and
          only if one element is a multiple of the other.

          The proof is easy.
          A set $\set{\vec{v}_1,\vec{v}_2}$ is linearly dependent if and only
          if there is a relationship $c_1\vec{v}_1+c_2\vec{v}_2=\zero$ 
          with either $c_1\neq 0$ or $c_2\neq 0$ (or both).
          That holds if and only if $\vec{v}_1=(-c_2/c_1)\vec{v}_2$
          or $\vec{v}_2=(-c_1/c_2)\vec{v}_1$ (or both).

\end{Answer}
