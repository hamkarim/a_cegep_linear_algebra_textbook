%-----------------------------------------------------
% index key words
%-----------------------------------------------------
\index{linear independence}
\index{linear dependence}

%-----------------------------------------------------
% name, leave blank
% title, if the exercise has a name i.e. Hilbert's matrix
% difficulty = n, where n is the number of stars
% origin = "\cite{ref}"
%-----------------------------------------------------
\begin{Exercise}[
name={},
title={}, 
difficulty=0,
origin={\cite{JH}}]
     With a some calculation we can get formulas to determine whether or
     not a set of vectors is linearly independent. 
\Question Show that this subset of \( \Re^2 \)
         \begin{equation*}
           \set{\colvec{a \\ c},\colvec{b \\ d}}
         \end{equation*}
         is linearly independent if and only if \( ad-bc\neq 0 \).
\Question Show that this subset of \( \Re^3 \)
         \begin{equation*}
           \set{\colvec{a \\ d \\ g},
                \colvec{b \\ e \\ h},
                \colvec{c \\ f \\ i}  }
         \end{equation*}
         is linearly independent iff
         \( aei+bfg+cdh-hfa-idb-gec \neq 0 \).
\Question When is this subset of \( \Re^3 \)
         \begin{equation*}
           \set{\colvec{a \\ d \\ g},
                \colvec{b \\ e \\ h} }
         \end{equation*}
         linearly independent?

\end{Exercise}

\begin{Answer}
\Question Assuming first that \( a\neq 0 \),
          \begin{equation*}
            x\colvec{a \\ c}
            +y\colvec{b \\ d}
            =\colvec{0 \\ 0}
          \end{equation*}
          gives
          \begin{equation*}
            \begin{linsys}{2}
               ax  &+  &by &=  &0  \\
               cx  &+  &dy &=  &0  
            \end{linsys}
	\end{equation*}
	and Gaussian elimination
        \begin{equation*}    
	\begin{linsys}{2}
               ax  &+  &by           &=  &0  \\
                   &   &(-(c/a)b+d)y &=  &0  
             \end{linsys}
          \end{equation*}
          which has a solution if and only if
          \( 0\neq-(c/a)b+d=(-cb+ad)/d \)
          (we've assumed in this case that \( a\neq 0 \), and so 
          back substitution yields a unique solution).

          The \( a=0 \) case is also not hard break it into the 
          \( c\neq 0 \) and \( c=0 \) subcases and 
          note that in these cases \( ad-bc=0\cdot d-bc \).
        \Question The equation
          \begin{equation*}
            c_1\colvec{a \\ d \\ g}
            +c_2\colvec{b \\ e \\ h}
            +c_3\colvec{c \\ f \\ i}
            =\colvec{0 \\ 0 \\ 0}
          \end{equation*}
         expresses a homogeneous linear system.
         We proceed by writing it in matrix form and applying Gauss's Method.

         We first reduce the matrix to upper-triangular.
         Assume that \( a\neq 0 \).
         With that, we can clear down the first column.
         \begin{equation*}
           \begin{amat}{3}
              1   &b/a           &c/a        &0   \\
              0   &(ae-bd)/a     &(af-cd)/a  &0   \\
              0   &(ah-bg)/a     &(ai-cg)/a  &0
            \end{amat}                                            
         \end{equation*}
         Then we get a $1$ in the second row, second column entry.
         (Assuming for the moment that \( ae-bd\neq 0 \), in order
         to do the row reduction step.)
         \begin{equation*}
           \begin{amat}{3}
              1   &b/a           &c/a             &0  \\
              0   &1             &(af-cd)/(ae-bd) &0  \\
              0   &(ah-bg)/a     &(ai-cg)/a       &0
            \end{amat}
         \end{equation*}
         Then, under the assumptions, we perform
         the row operation $((ah-bg)/a)\rho_2+\rho_3$
         to get this.
         \begin{equation*}
           \begin{amat}{3}
              1   &b/a   &c/a                              &0 \\
              0   &1     &(af-cd)/(ae-bd)                   &0 \\
              0   &0     &(aei+bgf+cdh-hfa-idb-gec)/(ae-bd) &0
            \end{amat}
         \end{equation*}
         Therefore, the original system is nonsingular
         if and only if the above \( 3,3 \) entry is nonzero
         (this fraction is defined because of the \( ae-bd\neq 0 \) assumption).
         It equals zero if and only if the numerator is zero. 

         We next worry about the assumptions.
         First, if \( a\neq 0 \) but \( ae-bd=0 \) then we swap row 2 and row 3
         \begin{multline*}
           \begin{amat}{3}
              1   &b/a           &c/a        &0   \\
              0   &(ah-bg)/a     &(ai-cg)/a  &0   \\
              0   &0             &(af-cd)/a  &0
            \end{amat}
         \end{multline*}
         and conclude that the system is nonsingular if and only if either
         \( ah-bg=0 \) or \( af-cd=0 \).
         That's the same as asking that their product be zero:
         \begin{align*}
            ahaf-ahcd-bgaf+bgcd
            &=0                   \\
            ahaf-ahcd-bgaf+aegc
            &=0                   \\
            a(haf-hcd-bgf+egc)
            &=0
         \end{align*}
         (in going from the first line to the second we've applied the
         case assumption that $ae-bd=0$ by substituting $ae$ for $bd$).
         Since we are assuming that \( a\neq 0 \), 
         we have that \( haf-hcd-bgf+egc=0 \).
         With $ae-bd=0$ we can rewrite this to fit the form we need:~in
         this \( a\neq 0 \) and \( ae-bd=0 \) case, the given system
         is nonsingular when
         \( haf-hcd-bgf+egc-i(ae-bd)=0 \), as required.

         The remaining cases have the same character.
         Do the \( a=0 \) but \( d\neq 0 \) case and the \( a=0 \) and
         \( d=0 \) but \( g\neq 0 \) case by first swapping rows and
         then going on as above.
         The \( a=0 \), \( d=0 \), and \( g=0 \) case is easy a set with a
         zero vector is linearly dependent, and the formula comes out
         to equal zero.
       \Question It is linearly dependent if and only if either vector is a
         multiple of the other.
         That is, it is not independent iff
         \begin{equation*}
           \colvec{a \\ d \\ g}=r\cdot\colvec{b \\ e \\ h}
           \quad\text{or}\quad
           \colvec{b \\ e \\ h}=s\cdot\colvec{a \\ d \\ g}
         \end{equation*}
         (or both) for some scalars $r$ and $s$.
         Eliminating $r$ and $s$ in order to restate this condition only in
         terms of the given letters $a$, $b$, $d$, $e$, $g$, $h$, we have that 
         it is not independent, it is dependent iff
         \( ae-bd=ah-gb=dh-ge \)
\end{Answer}
