%-----------------------------------------------------
% index key words
%-----------------------------------------------------
\index{polynomial space}
\index{subspace}
\index{vector space!subspace}


%-----------------------------------------------------
% name, leave blank
% title, if the exercise has a name i.e. Hilbert's matrix
% difficulty = n, where n is the number of stars
% origin = "\cite{ref}"
%-----------------------------------------------------
\begin{Exercise}[
name={},
title={}, 
difficulty=0,
origin={\cite{JH}}]
Is \( \Re^2 \) a subspace of \( \Re^3 \)?

\end{Exercise}

\begin{Answer}
No.
      Subspaces of \( \Re^3 \) are sets of three-tall vectors, while
      \( \Re^2 \) is a set of two-tall vectors.
      Clearly though, \( \Re^2 \) is ``just like'' this subspace of 
      \( \Re^3 \).
      \begin{equation*}
        \set{\colvec{x \\ y \\ 0}\suchthat x,y\in\Re}
      \end{equation*}  
\end{Answer}
