%-----------------------------------------------------
% index key words
%-----------------------------------------------------
\index{subspace}
\index{vector space!subspace}


%-----------------------------------------------------
% name, leave blank
% title, if the exercise has a name i.e. Hilbert's matrix
% difficulty = n, where n is the number of stars
% origin = "\cite{ref}"
%-----------------------------------------------------
\begin{Exercise}[
name={},
title={}, 
difficulty=0,
origin={\cite{JH}}]
$\Re^3$ has infinitely many subspaces.
Do every nontrivial space have infinitely many subspaces?
\end{Exercise}

\begin{Answer}
 No.
      The only subspaces of \( \Re^1 \) are the space itself and its 
      trivial subspace.
      Any subspace $S$ of $\Re$ that contains a nonzero member $\vec{v}$ 
      must contain the set of all of its scalar multiples 
      $\set{r\cdot\vec{v}\suchthat r\in\Re}$. 
      But this set is all of $\Re$.  


\end{Answer}
