%-----------------------------------------------------
% index key words
%-----------------------------------------------------
\index{norm}
\index{unit vector}

%-----------------------------------------------------
% name, leave blank
% title, if the exercise has a name i.e. Hilbert's matrix
% difficulty = n, where n is the number of stars
% origin = "\cite{ref}"
%-----------------------------------------------------
\begin{Exercise}[
name={},
title={}, 
difficulty=0,
origin={\cite{JH}}]
    Show that if \( r\geq 0 \) then
     \( r\vec{v} \) is \( r \) times as long
     as \( \vec{v} \).
     What if \( r< 0 \)?
\end{Exercise}

\begin{Answer}
      For the first question, assume that \( \vec{v}\in\Re^n \) and
      \( r\geq 0 \), take the root, and factor.
      \begin{equation*}
        \norm{r\vec{v}\,}
        =\sqrt{(rv_1)^2+\cdots+(rv_n)^2}     
        =\sqrt{r^2({v_1}^2+\cdots+{v_n}^2}     
        =r\norm{\vec{v}\,}
      \end{equation*}
      For the second question, the result is \( r \) times as long, but it
      points in the opposite direction in that
      \( r\vec{v}+(-r)\vec{v}=\zero \). 
\end{Answer}
