%-----------------------------------------------------
% index key words
%-----------------------------------------------------
\index{dot product}
\index{algebraic properties}

%-----------------------------------------------------
% name, leave blank
% title, if the exercise has a name i.e. Hilbert's matrix
% difficulty = n, where n is the number of stars
% origin = "\cite{ref}"
%-----------------------------------------------------
\begin{Exercise}[
name={},
title={}, 
difficulty=0,
origin={\cite{JH}}]
    Describe the algebraic properties of dot product.
\Question Is it right-distributive over addition:
        \(
           (\vec{u}+\vec{v})\dotprod\vec{w}
           =
           \vec{u}\dotprod\vec{w}+\vec{v}\dotprod\vec{w} \)?
\Question Is it left-distributive (over addition)?
\Question Does it commute?
\Question Associate?
\Question How does it interact with scalar multiplication?
\end{Exercise}

\begin{Answer}
      In each item below, assume that the vectors 
      \( \vec{u},\vec{v},\vec{w}\in\Re^n \) 
      have components
      \( u_1,\ldots,u_n,v_1,\ldots,w_n \).
\Question Dot product is right-distributive.
           \begin{align*}
              (\vec{u}+\vec{v})\dotprod\vec{w}
              &=[\colvec{u_1 \\ \vdotswithin{u_1} \\ u_n}
                +\colvec{v_1 \\ \vdotswithin{v_1} \\ v_n}]\dotprod
                \colvec{w_1 \\ \vdotswithin{w_1} \\ w_n}               \\
              &=
              \colvec{u_1+v_1 \\ \vdotswithin{u_1+v_1} \\ u_n+v_n}\dotprod
                \colvec{w_1 \\ \vdotswithin{w_1} \\ w_n}               \\
              &=
              (u_1+v_1)w_1+\cdots+(u_n+v_n)w_n              \\
              &=
              (u_1w_1+\cdots+u_nw_n)+(v_1w_1+\cdots+v_nw_n)  \\
              &=
              \vec{u}\dotprod\vec{w}+\vec{v}\dotprod\vec{w}
           \end{align*}
\Question Dot product is also left distributive:
          $\vec{w}\dotprod(\vec{u}+\vec{v})=
              \vec{w}\dotprod\vec{u}+\vec{w}\dotprod\vec{v}$.
          The proof is just like the prior one.
\Question Dot product commutes.
          \begin{equation*}
            \colvec{u_1 \\ \vdotswithin{u_1} \\ u_n}\dotprod
              \colvec{v_1 \\ \vdotswithin{v_1} \\ v_n}
            =u_1v_1+\cdots+u_nv_n   
            =v_1u_1+\cdots+v_nu_n   
            =\colvec{v_1 \\ \vdotswithin{v_1} \\ v_n}\dotprod
              \colvec{u_1 \\ \vdotswithin{u_1} \\ u_n}
          \end{equation*}
\Question Because \( \vec{u}\dotprod\vec{v} \) 
          is a scalar, not a vector,
          the expression \( (\vec{u}\dotprod\vec{v})\dotprod\vec{w} \) makes no
          sense; the dot product of a scalar and a vector is not defined.
\Question This is a vague question so it has many answers.
          Some are 
          (1)~\( k(\vec{u}\dotprod\vec{v})=(k\vec{u})\dotprod\vec{v} \)
          and \( k(\vec{u}\dotprod\vec{v})=\vec{u}\dotprod(k\vec{v}) \),
          (2)~\( k(\vec{u}\dotprod\vec{v})\neq (k\vec{u})\dotprod(k\vec{v}) \)
          (in general; an example is easy to produce), and
          (3)~\( \absval{k\vec{v}\,}=\lvert k\rvert\absval{\vec{v}\,} \) 
          (the connection between
          length and dot product is that the square of the length is the
          dot product of a vector with itself).
\end{Answer}
