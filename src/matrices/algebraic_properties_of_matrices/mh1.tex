%-----------------------------------------------------
% index key words
%-----------------------------------------------------
\index{matrix!involutory}
\index{matrix!transpose}

%-----------------------------------------------------
% name, leave blank
% title, if the exercise has a name i.e. Hilbert's matrix
% difficulty = n, where n is the number of stars
% origin = "\cite{ref}"
%-----------------------------------------------------
\begin{Exercise}[
name={},
title={}, 
difficulty=0,
origin={\cite{MH}}]
An \emph{involutory matrix} is a matrix $A$ such that $A^2=I$.
\Question Show that $I_n$ is involutory.
\Question Show that if $A$ is an involutory matrix , then $A$ is  a square matrix.
\Question If $A$ is involutory, is $A^T$ also involutory? If yes, prove it. If no, find a counter example.
\Question Suppose that $A$ and $B$ are involutory,
Is $AB$ also involutory? If yes, prove it. If no, find a condition on $A$ and $B$ so that $AB$ is also involutory?
\end{Exercise}
\begin{Answer}
\Question Involutory since $I^2=I$
\Question For $A^2$ to be defined, $A$ needs to have the same number of rows as columns.
\Question Hint: Show that $(\trans(A))^2=I$.
\Question $AB$ is inovolutory under the condition that $A$ and $B$ commute.
\end{Answer}
